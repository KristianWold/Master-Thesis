%==========================================================
%------------------------- misc ---------------------------
%==========================================================

%-------- useful ----------
“ ”       
“”

%--------- misc -----------

% Chemformula
\ch{3 H2O}

%----------------- TODO -------------------
% beside text
\todo{Is it possible to add a subsubparagraph?}
% inline of text
\todo[inline,color=green]{I think that a summary of this exciting chapter should be added.}

%-------------- framed environment ---------------
% mdframed
\begin{mdframed}[style=MyFrame]
    $E = m c^2$
\end{mdframed}

% tcolorbox
% default: red frame
\begin{equation*}
\tcbhighmath[drop fuzzy shadow=blue]
{
    \begin{aligned}
        E = m c^2
    \end{aligned}
}
\end{equation*} 

% with blue frame
\begin{equation*}
\tcbhighmath[colframe=blue,drop fuzzy shadow=black]
{
    \begin{aligned}
        E = m c^2
    \end{aligned}
}
\end{equation*}

%------------- cross-referencing -----------------
You can redefine the prefix for the autoref command. For instance for the sections use
\def\sectionautorefname{New prefix for the sections}
More generally
\def\<type>autorefname{<new name>}
Followed by
\addto\extrasenglish{%
  \renewcommand{\sectionautorefname}{Section}%
}

%href problem:
%\section{blabla $x$}
%gives error msg:
%token not allowed in a pdf string
%solution:
%\section{blabla \texorpdfstring{$x$}{Lg}}



%------------------ algorithm ---------------------

\alglanguage{pseudocode}  % or \alglanguage{pascal}
\begin{algorithm}[H]
\caption{<your caption for this algorithm>}
\label{<your label for references later in your document>}
\begin{algorithmic}[1]  %[1] no. each line, [5] each 5th etc
<algorithmic environment>
\end{algorithmic}
\end{algorithm}