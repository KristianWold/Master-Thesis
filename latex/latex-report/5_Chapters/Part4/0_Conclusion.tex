%================================================================
\chapter{Conclusion \& Future Research}\label{chap:Conclusion}
%================================================================

%----------------------------------------------------------------
\section{Conclusion}\label{sec:conclusion}
%----------------------------------------------------------------


%----------------------------------------------------------------
\section{Future Research}\label{sec:future}
%----------------------------------------------------------------


%----------------------------------------------------------------
\section{Todo}\label{sec:todo}
%----------------------------------------------------------------

\begin{itemize}
    \item Noise simulation only an approximation
    \item Classical activation saturation(tanh, xavier paper). Plateau, akin to concentration of PQC output around mean.  
    \item More details on vanishing PQC gradient
    \item Calculate complexity of QCN
    \item batch norm for QCN
    \item fewer parameters, relieving bottleneck for extremly large models. 
    \item is QCN scalable for many qubits? With log n depth and local observable? QCN perfect usecase!
    \item is QCN with few qubits smart? Scales no better than linearly. We already have very powerful classical computers. 
    \item more details FIM
    \item Elaborate on other ways of overcoming vanishing gradient(layerwise learning)
    \item Santiago quantum computer
    \item Write about transpiler
    \item Source of non-linearity, measurement (Shuld vanilla QNN). Also, imprivitive 2-qubit gate, entanglement, important for ansatz. Need to motive that QML can be used on NISQ! No amplification of defect in a single quantum circuit. Property of PQC: learn to compensate for hardware specific faults. 
\end{itemize}