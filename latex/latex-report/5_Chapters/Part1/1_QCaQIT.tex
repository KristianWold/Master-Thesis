%================================================================
\chapter{Quantum Computing}\label{chap:QuantumComputing}
%================================================================
This chapter introduces the fundamentals of quantum computing. The content of this chapter is mainly based on material in \cite{NielsenQuantum}.


%================================================================
\section{States in Quantum Mechanics}\label{sec:IntroQM}
%===============================================================
In quantum mechanics, isolated physical systems are completely described by its \emph{state vector}, which lives in a complex vector space. In this thesis, we will be concerned with finite vector spaces $\mathbb{C}^n$, where the state is a n-tuple of complex numbers $(z_1, \cdots, z_n)$ called \emph{amplitudes}. Adopting Dirac notation, a state is denoted as 
\begin{equation}
    \ket{\psi} \sim \begin{bmatrix}
           z_{1} \\
           \vdots \\
           z_{n}
         \end{bmatrix},
\end{equation}
where $\psi$ is the label of the state, and $\ket{\cdot}$ indicates that it is a vector. More specifically, in quantum mechanics, the states live in \emph{Hilbert spaces}, which is a vector space that has a well-defined inner product. The inner product of two states $\ket{\psi}, \ket{\psi'} \in \mathbb{C}^n$ is denoted 

\begin{equation}
    \braket{\psi'}{\psi} \equiv [z_1'^*, \cdots, z_n'^*] 
    \begin{bmatrix}
        z_{1} \\
        \vdots \\
        z_{n}
    \end{bmatrix}
    = \sum_{i=1}^{n} z_i'^* z_i, 
\end{equation}
where $z^*$ indicates the complex conjugate, i.e. if $z = a + ib$, we have that $z^* = a - ib$. As a constraint on the amplitudes, state vectors describing physical systems are of unit norm, meaning 

\begin{equation}
    \braket{\psi}{\psi} \equiv [z_1^*, \cdots, z_n^*] 
    \begin{bmatrix}
        z_{1} \\
        \vdots \\
        z_{n}
    \end{bmatrix}
    = \sum_{i=1}^{n} |z_i|^2 = 1.
\end{equation}

%================================================================
\subsection{The Qubit}\label{sec:TheQubit}
%===============================================================
As is common in quantum computing, we will be focusing on perhaps the simplest possible quantum system, the \emph{qubit}, which is a two-level system defined on $\mathbb{C}^2$.  There are several ways of implementing qubits in hardware, of which some will be discussed later, although  the specific physical realization is not necessary to account for when discussing quantum computing. In abstract terms, the state of a qubit can be formulated as

\begin{equation}
\ket{\psi} = \alpha \ket{0} + \beta \ket{1},
\end{equation}
where $\alpha$ and $\beta$ are complex numbers, and $\ket{0}$ and $\ket{1}$ are orthonormal states known as the \emph{computational basis states} and are specially defined by the implementation of the hardware. This linear combination of states is a very important principle of quantum mechanics and is called \emph{superposition}; the system is in neither state $\ket{0}$ nor $\ket{1}$, but both at the same time(unless either $\alpha$ or $\beta$ is zero). In general, if states $\ket{\psi}$ and $\ket{\phi}$ are allowed, then so is also the linear combination $\alpha \ket{\psi} + \beta \ket{\phi}$, where $|\alpha|^2 + |\beta|^2 = 1$.

Being the "atom" of quantum computing, the qubit is very reminiscent of the classical bit, which is always definitely "0" or "1". However, as we have seen, the qubit also may assume any normalized linear combination of the two states.
%================================================================
\subsection{Multiple Qubits}\label{sec:Multiple Qubits}
%===============================================================
As a central property of quantum mechanics, it is possible to create composit systems by combining several smaller quantum systems. This can be used to construct systems of multiple qubits, whose state can be expressed, if the qubits are independent, as

\begin{equation}
\ket{\psi_1 \psi_2 \cdots \psi_n} \equiv \ket{\psi_1} \otimes \ket{\psi_2} \otimes \cdots \ket{\psi_n}.
\end{equation}
Here, the tensor product "$\otimes$" was used to indicate that each state $\ket{\psi_i}$ lives in its own $\mathbb{C}^2$ space. Using the principle of superpostion, one may make a linear combination of several multi-qubit states, where each $\ket{\psi_i}$ is either $\ket{0}$ or $\ket{1}$. In general, this can be written as

\begin{equation}\label{eq:MultiQubitState}
\ket{\psi} = \sum_{\boldsymbol{v}}c_{\boldsymbol{v}}\ket{\boldsymbol{v}_1} \otimes \ket{\boldsymbol{v}_2} \otimes \cdots \ket{\boldsymbol{v}_n},
\end{equation}
where $\boldsymbol{v} \in \{0,1\}^n$ sums over all possible binary strings of length $n$. As there are $2^n$ unique strings, we arrive at the remarkable result that one also needs $2^n$ amplitudes $c_{\boldsymbol{v}}$ to describe the state of $n$ qubits in general. In other words, the information possible to store in the quantum state of $n$ qubits in exponential in $n$, as opposed to the linear information of a equivalent classical system of classical bits. In a sense, the quantum information is "bigger" than the classical information. This is a very interesting property when discussing capabilities of quantum computing, which we will return to when discussing the usefulness of quantum computing in relation to machine learning.

%================================================================
\subsection{Measuring Qubits}\label{sec:MeasuringState}
%===============================================================
It appears the information encoded in quantum systems are much greater than the information in a corresponding classical system, at least in the case of qubits versus bits. How can one interact with this information? Unlike classical bits, whose state can always be measured exactly, the state of one or multiple qubits can not be measured and determined. Returning to the single qubit example, one can choose to perform a measurement in the computational basis on a qubit in the state $\ket{\psi} = \alpha \ket{0} + \beta \ket{1}$. Resulting from the measurement will be \emph{either} $\ket{0}$ \emph{or} $\ket{1}$, with probability $|\alpha|^2$ and $|\beta|^2$, respectively. For multiple qubits in a general state \autoref{eq:MultiQubitState}, a measurement on all qubits will grant a state in the computational basis, i.e. $\ket{\boldsymbol{v}_1} \otimes \ket{\boldsymbol{v}_2} \otimes \cdots \ket{\boldsymbol{v}_n}$ for some binary string $\boldsymbol{v} \in \{0,1\}^n$, with probability $|c_{\boldsymbol{v}}|^2$. This motivates why states in qunatum mechanics needs to have unity norm, i.e
\begin{equation}\label{eq:MultiQubitState}
\sum_{\boldsymbol{v}}|c_{\boldsymbol{v}}|^2 = 1,
\end{equation} 
as the probabilities of any outcome must sum to $1$.


%================================================================
\section{Quantum Circuits}\label{sec:QuantumCircuits}
%===============================================================
We have until now looked into how quantum states can encode information, and how information can be interacted with via measurement. How then can quantum mechanics be used for computation? To do computations, it is necessary to introduce some dynamical transformation of the quantum state. In quantum machanics, transformations can be formulated as 

\begin{equation}\label{eq:UnitaryEvolution}
 \ket{\phi} = U\ket{\psi},
\end{equation}
where $U$ is a \emph{unitary} operator that acts on the vector space where $\ket{\psi}$ and $\ket{\phi}$ lives. Unitary means that the operator $U$ is linear with the property that $U^{\dagger} = U^{-1}$, that is, the hermitian conjugate is equal to its inverse. This is a necessary property of linear operators in quantum mechanics as to ensure that the state stays normalized to 1:

\begin{equation}\label{eq:UnitaryEvolution}
 \braket{\phi}{\phi} = \bra{\psi}U^{\dagger}U\ket{\psi} = \bra{\psi}\ket{\psi} =1.
\end{equation}
Assuming $\ket{\psi}$ is initially normalized, so is $\ket{\phi}$ after a unitary transformation.

By construction, quantum computers allows for the application of carefully selected sequences of operators that transforms the state in a desired way, often called a \emph{quantum circuit}. Typical operators used in quantum computing, often called quantum gates, act on either on one or multiple qubits, and bear resemblance to logical operations in the classical context. Here, we will present the most important quantum gates.

%================================================================
\subsection{Single Qubit Operations}\label{sec:ControlledOperations}
%===============================================================



%================================================================
\subsection{Controlled Operations}\label{sec:ControlledOperations}
%===============================================================


%================================================================
\subsection{Expectation Values}\label{sec:ExpectationValues}
%===============================================================




%================================================================
\section{Quantum Computers}\label{sec:QuantumCircuits}
%===============================================================



