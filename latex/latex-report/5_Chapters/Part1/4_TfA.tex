%================================================================
\chapter{Tools for Analysis}\label{chap:TfA}
%================================================================



%================================================================
\section{Trainability}\label{sec:Trainability}
%================================================================
In machine learning, \emph{trainability} refers to how easily a particular model can be trained under different conditions \cite{abbas2020power}. A common way of exploring the trainability is by exploring the geometry of the loss landscape. for example, the loss function of dense neural networks exhibit local flatness for most directions in parameter space, and strong distortion in others \cite{karakida2019universal}. In a loss landscape that is mostly flat, the gradient of the model will tend to diminish, known as \emph{vanishing gradient}, making it hard to train the model using gradient-based methods. This is also known to worsen with the number of layers, making the training of deep models prohibitive.

To investigate the flatness and distortions of the loss landscape, a common metric to use is the \emph{hessian} of the loss, which we will introduce in the next subsection. 
%================================================================
\subsection{Hessian Matrix}\label{sec:HessianMatrix}
%================================================================
Let $f(\boldsymbol{x}^{(k)}; \boldsymbol{\theta})$ be a parameterized and differentiable model, where $\boldsymbol{x}^{(k)} \in \mathbb{R}^p$ is $p$ features, and $\boldsymbol{\theta} \in \mathbb{R}^{n_\theta}$ is $n_{\theta}$ model parameters. For a general loss function on the form \autoref{eq:LossFunction}, $L(\boldsymbol{\theta}) = \sum{_{k=1}^{N}L(f(\boldsymbol{x}^{(k)};\boldsymbol{\theta}), y^{(k)}})$, where $N$ is the number of samples in the data set, the hessian matrix of the loss function is given by

\begin{equation}
\label{eq:Hessian}
    H_{ij} = \frac{\partial^2 L(\boldsymbol{\theta})}{\partial \boldsymbol{\theta}_i\partial \boldsymbol{\theta}_j}.
\end{equation}

The hessian matrix is a $n_\theta \times n_\theta$ matrix that quantifies the curvature of the loss function locally in the parameter space at the point of $\boldsymbol{\theta}$. This is a much-studied quantity in the machine learning community, and it has been used to study the loss landscape both for classical and quantum mechanical machine learning models \cite{LeCun2012} \cite{Huembeli_2021}. In particular, it's eigenvalue spectrum quantifies the amount of curvature in various directions. Typical for classical neural networks, the spectrum is characterized by the presence of many eigenvalues near zero, with the exception of a few large(so called "big killers"). This indicated the that loss landscape is mostly flat, with huge distortions in a few directions, which in turn causes slow optimization as discussed earlier. 

%================================================================
\subsection{Empirical Fisher Information Matrix}\label{sec:EFIM}
%================================================================
An apparent shortcoming of the hessian matrix \autoref{eq:Hessian} is the huge computational cost of computing it, requiring the evaluation of $\mathcal{O}(n_\theta^2)$ double derivatives. This is particularly expensive for models of many parameters, which e.g. neural networks tend to be. An alternative and related quantity, called the \emph(Empirical Fisher Information Matrix)(EFIM), can be calculated using $\mathcal{O}(n_\theta)$ first order derivatives, which is much more suited for big models. We will now derive the EFIM and relate it to the hessian.

Assume a square loss $\frac{1}{2N}\sum_{k=1}^{N} (f(\boldsymbol{x}^{(k)}; \boldsymbol{\theta}) - y^{(k)})^2$. Computing \autoref{eq:Hessian} with this loss results in 
    
\begin{equation}\label{eq:HessianSquareLoss}
    H_{ij} = F_{ij} -
    \frac{1}{N}\sum_{k=1}^{N} (y^{(k)} - f(\boldsymbol{x}^{(k)};\boldsymbol{\theta}))\frac{\partial^2 f(\boldsymbol{x}^{(k)};\boldsymbol{\theta})}{\partial \boldsymbol{\theta}_i\partial \boldsymbol{\theta}_j}, 
\end{equation}
where F is given by
\begin{equation}
\label{eq:EmpiricalFisher}
    F_{ij} =  \frac{1}{N}\sum_{k=1}^{N}
    \frac{\partial f(\boldsymbol{x}_k;\boldsymbol{\theta})}{\partial \boldsymbol{\theta}_i}
    \frac{\partial f(\boldsymbol{x}_k;\boldsymbol{\theta})}{\partial \boldsymbol{\theta}_j}.
\end{equation}.

The quantity F is often called the empirical fisher information matrix(EFIM)\cite{karakida2019universal}, and ulike the hessian matrix, it can be calculated only using first derivatives, i.e. the gradient.
From \autoref{eq:HessianSquareLoss} it can been seen to coincide with the hessian matrix if $f(\boldsymbol{x}^{(k)};\boldsymbol{\theta}) = y^{(k)}$, since the terms in the last sum vanishes. This is the case if the model manage to perfectly replicate the targets from the inputs, which is approximately true for well-trained models what fit the data sufficiently. However, even for untrained models, the EFIM is sometimes used as a cheaper alternative to the hessian matrix, particularly for investigating the geometry of the loss landscape via its eigenvalue spectrum. This has been done both for classical and quantum mechanical ML models \cite{karakida2019universal} \cite{abbas2020power}. Although these investigations, as well as this thesis, use the EFIM for untrained models, and so does not coincide with the hessian matrix, it still serves as a heuristic for addressing the flatness and distortions of the loss landscape. 

%================================================================
\section{Expressivity}\label{sec:Expressivity}
%================================================================


%================================================================
\subsection{Trajectory Length}\label{sec:TrajectoryLength}
%================================================================

